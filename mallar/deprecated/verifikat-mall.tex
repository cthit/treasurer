\documentclass[7pt,a4paper]{article}
\usepackage[utf8]{inputenc}
%\usepackage[T1]{fontenc}
\usepackage[swedish]{babel}
\pagenumbering{arabic}
\usepackage{graphicx}
\usepackage{fancyhdr}
\usepackage{fullpage}
\usepackage{array}
\usepackage{ifpdf}
\usepackage[left=20mm,right=20mm,top=10mm,bottom=20mm]{geometry}
\usepackage[yyyymmdd]{datetime} %Ger mer valfrihet kring datumformat
\usepackage{url}
\usepackage{multirow}
%\usepackage[normaalem]{ulem}

\graphicspath{ {images/} }


%Veckodagar på svenska
\providecommand*{\dayofweeknameidswedish}[1]{%
\ifcase#1\relax
\or söndagen%
\or måndagen%
\or tisdagen%
\or onsdagen%
\or torsdagen%
\or fredagen%
\or lördagen%
\fi}


%%%% Fina headers är bra skit. %%%%
\pagestyle{fancy}
\headheight 35pt
\headwidth \textwidth
\addtolength{\textwidth}{-65pt}
\headsep 40pt
\addtolength{\textheight}{-65pt}
\renewcommand{\headrulewidth}{0pt}
%%%% Välj logga (styretlogo, sektionslogo eller din funktionärslogo) och ändra
\fancyhead[L]{\includegraphics[height=3\baselineskip]{sektionslogo.png}}
\fancyhead[R]{Ver. nr \hspace{2cm}}
%\thispagestyle{empty}
\pagenumbering{gobble}
%\fancyfoot[L]{Revisor \\ Anna Rosenberg \\ annros@student.chalmers.se}
%\fancyfoot[C]{\url{http://www.ftek.chalmers.se/revisor} \\ revisor@ftek.chalmers.se}
%\fancyfoot[R]{Revisorssuppleant \\ Marcus Birgersson \\ marbirg@student.chalmers.se}


%%%%%%%EGNA KOMMANDON %%%%%%%%%%%%%%
\newcommand{\signbox}[3]{

\vspace{2cm}
#1  \\
\vspace{5mm}\\
\line(1,0){150}\\
 #2 \\ #3\newpage}

\newdateformat{mydate}{\dayofweekname{\THEDAY}{\THEMONTH}{\THEYEAR} den \THEDAY :a \monthname[\THEMONTH] \THEYEAR}%END




%%%% Här börjar det riktiga dokumentet %%%%

\begin{document}

\renewcommand{\dateseparator}{-} %Omdefinierar hur datum visas
\renewcommand{\arraystretch}{2.0}

\vspace{1cm}



\textbf{Arrangemangsinformation:} \\

\begin{tabular}{|p{10.05cm}|p{6cm}|}
\hline
Arrangemang: & Datum:
\end{tabular}

\begin{tabular}{|p{16.5cm}|}
\hline
Förklarande text:   
\\ \cline{1-1}
\\ \cline{1-1}\\ \hline
\end{tabular} \\

\textit{Bifoga kvitto på baksidan} \\

\vspace{1cm}

\textbf{Verifikatinformation:} \\

\begin{tabular}{|p{0.57\linewidth}|p{0.1\linewidth}|p{0.05\linewidth}|p{0.14\linewidth}|p{0.14\linewidth}|}
\hline
 Kontonamn & Kontonr & Ks & Debet & Kredit \\ \cline{1-5}
 & & & &
 \\ \cline{1-5}
 & & & &
 \\ \cline{1-5}
 & & & &
 \\ \cline{1-5}
 & & & &
 \\ \cline{1-5}
 & & & &
 \\ \cline{1-5}
 & & & &
 \\ \cline{1-5}
 \multicolumn{5}{|l|}{Totalbelopp:} \\ \cline{1-5}
\hline
\end{tabular}


\vspace{5cm}

%\textbf{Information från inköpare} \\

%\begin{tabular}{|p{3cm}|p{10cm}|p{3cm}|}
%\hline
%Datum: & Namn: & Belopp: \\ \hline
%\end{tabular}

%\begin{tabular}{|p{5cm}|p{5cm}|p{6cm}|}
%Bank: & Clearing nr: & Konto nr: \\ \hline
%\end{tabular}
%\vspace{1cm}

\textbf{Att fyllas i av båda parter:} \\

\begin{tabular}{|p{8cm}|p{8cm}|}
\hline
Underskrift: & Attest: \\ \hline
\end{tabular}

\begin{tabular}{|p{8cm}|p{8cm}|}
Namnförtydligande: & Namnförtydligande:  \\ \hline
\end{tabular}\\

\textit{Notera att kassören inte får attestera sina egnenhändiga inköp} \\


%\signbox{d}{Marcus Birgersson}{Revisorssuppleant}


\end{document}
